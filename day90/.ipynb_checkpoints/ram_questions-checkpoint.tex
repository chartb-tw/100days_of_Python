\documentclass{article}
\usepackage[utf8]{inputenc}

\usepackage{graphicx}
\usepackage{amsmath, amsfonts, amsthm, bbm}
\usepackage{mathtools}
\usepackage{ mathdots }
\usepackage{pgfplots}
\usepackage[all]{xy}
\pgfplotsset{width=\columnwidth,compat=1.13}
\usepackage{hyperref}
\usepackage{float}
\usepackage{titling}
\usepackage{caption}
\usepackage{subcaption}
\usepackage{cancel}
\usepackage[toc,page]{appendix}

\graphicspath{{images/}}

\numberwithin{equation}{section}

\theoremstyle{definition}
\newtheorem{definition}{Definition}

\newtheorem{theorem}{Theorem}
\newtheorem{lemma}{Lemma}
\newtheorem{prop}{Proposition}
\newtheorem{quest}{Question}

\newcommand{\half}{\frac{1}{2}}
\newcommand{\dbyd}[2]{\frac{\partial #1}{\partial #2}}
\newcommand{\vsig}{\vec{\sigma}}
\newcommand{\N}{\mathbb{N}}
\newcommand{\Z}{\mathbb{Z}}
\newcommand{\Q}{\mathbb{Q}}
\newcommand{\R}{\mathbb{R}}
\newcommand{\C}{\mathbb{C}}
\newcommand{\norm}[1]{\left\lVert #1 \right\rVert}
\newcommand{\comp}{\circ}
\renewcommand{\O}{\mathcal{O}}
\renewcommand{\u}{{u_1}}
\renewcommand{\v}{{u_2}}
\newcommand{\laplace}{\Delta}
\renewcommand{\line}[2]{{\overrightarrow{#1,#2}}}

\makeatletter
\makeatother

\newcommand{\todo}[1]{{\color{red}\textbf{TODO!: #1}}}

\setlength{\parindent}{0em}
\setlength{\parskip}{1em}

\renewcommand{\.}{\,.}


\title{Rings and modules - all questions}
\author{deez nutz $2k\infty$}
\date{\today}

\usepackage{geometry}
\geometry{a4paper, left=30mm, right=30mm, top=20mm, bottom=20mm}

\renewcommand{\baselinestretch}{1.5}

\begin{document}

%\maketitle

\begin{quest}[2018 A1]

\begin{itemize}
\item Define the terms \textit{ring, integral domain} and \textit{field}.
\item Is it true that every integral domain is a field? Justify your answer.
\end{itemize}
\end{quest}

\begin{quest}[2018 A2]
Suppose that $R$ is a ring and that $R[X]$ is a polynomial ring in one indeterminate over $R$. Suppose that $g\in R[X]$ is a monic element and that $f$ is any element of $R[X]$. Explain the process of \textit{long division} of $f$ by $g$. Illustrate your answer with the case where $R=\Z$, $g=X^2-5$ and $f=2X^3+1$.
\end{quest}

\begin{quest}[2018 A3]

\begin{itemize}
\item If $R,S$ are rings, define the notion of a \textit{homomorphism} from $R$ to $S$.
\item Define the notion of the \textit{kernel} of a homomorphism.
\item Define the notion of an \textit{ideal} of a ring $R$.
\item Is the set of all odd integers an ideal in the ring $\Z$? Justify your answer.
\end{itemize}
\end{quest}

\begin{quest}[2016 A1]

\begin{itemize}
\item Show that if
\begin{equation*}
I_1\subseteq I_2 \subseteq I_3 \subseteq \cdots
\end{equation*}
is an ascending chain of ideals, then $\bigcup_{n\geq 1} I_n$ is also an ideal of $A$.
\item Consider the ideals $(2)$ and $(3)$ in the ring $\Z$. Is $(2)\cup(3)$ also an ideal in $\Z$?
\end{itemize}
\end{quest}

\begin{quest}[2018 A4]
Construct a homomorphism $\phi:\Z[X]\to\C$ whose kernel equals $(X^2+5)$, the principal ideal generated by $X^2+5$. Justify your answer.
\end{quest}

\begin{quest}[2017 A1]
An element of a ring $R$ is said to be \textit{nilpotent} if there is an integer $m>0$ such that $x^m=0$. Show that if $x$ and $y$ are nilpotent elements of $A$, then $x+y$ is also nilpotent.
\end{quest}

\begin{quest}[2017 A2]
Is the polynomial ring $A[X]$ finitely generated as an $A$-module? Justify your answer.
\end{quest}

\begin{quest}[2017 A3]
State Eisenstien's criterion for the irreducibility of a polynomial over $\Q$. Use it to show that $X^3+3X+3$ is irreducible in $\Q[X]$. (You may use any other general result that you wish, provided you state it clearly.)
\end{quest}

\begin{quest}[2017 A4]
Define the notion of a \textit{prime ideal} in a ring $A$, and show that an ideal $I$ in $A$ is prime if and only if $A/I$ is an integral domain.
\end{quest}

\begin{quest}[2016 A4, 2015 A4 (Aug)]
Define the notion of a \textit{maximal ideal} in a ring $A$, and show that an ideal $I$ in $A$ is maximal if and only if $A/I$ is a field.
\end{quest}

\begin{quest}[2016 A2]
Show that for any ring $A$ and any element $a\in A$, the ring $A[X]/(X-a)$ is isomorphic to $A$.
\end{quest}

\begin{quest}[2015 A1 (both)]
Show that the rings $\Z[\sqrt{-7}]$ (May) and $\Z[\sqrt{-3}]$ (Aug) are not unique factorisation domains.
\end{quest}

\begin{quest}[2015 A3 (May)]
\begin{itemize}
\item Suppose that I is an ideal in the polynomial ring $A[X]$, that $f$ is a monic polynomial of $I$ of degree $n$ and that every nonzero element of $I$ has degree at least $n$. Show that $I=(f)$, the principal ideal generated by $f$.
\item Prove that $\Z[7^{1/3}]$ is isomorpic to $\Z[X]/(X^3-7)$, where $X$ is an indeterminate.
\end{itemize}
\end{quest}


% section B questions

\begin{quest}[combined 2018 B5, 2017 B5, 2015 A2 (May/Aug)]

\begin{itemize}
\item Define the notion of a \textit{Euclidean domain}.
\item Prove that $\Z$ and $\Z[i]$ are Euclidean.
\item Define the notion of a \textit{principal ideal domain} and show that every Euclidean domain is a principal ideal domain.
\end{itemize}
\end{quest}

\begin{quest}[2018 B6, 2015 A3 (Aug), 2016 B6]

\begin{itemize}
\item Define the notion of a \textit{Noetherian ring}. Prove that, if $R$ is a Noetherian ring, then so is the polynomial ring $R[X]$.
\item Consider the homomorphism $\phi : \Z[X] \to \R$ defined by $\phi(X)=\sqrt{5}$. Show that the kernel of $\phi$ is a principal ideal in $\Z[X]$ and find a generator of this ideal.
\end{itemize}
\end{quest}


\begin{quest}[combined 2018 B7, 2015 B5 (Aug)]

\begin{itemize}
\item Suppose that $R$ is a Noetherian ring and that $M$ is a finitely generated $R$-module. Explain how $M$ can be described in terms of matrices with entries in $R$. (You may assume that every submodule of a finitely generated $R$-module is also finitely generated.)
\item State the structure theorem for finitely generated modules over a Euclidean domain $R$ and explain how it can be derived from a theorem about matrices with entries in $R$.
\item Apply the structure theorem to the $\Z$-module $M$ generated by $m_1, m_2, m_3$ subject to the relations $m_1+2m_2+3m_3=0$ and $4m_1+5m_2+6m_3=0$.
\end{itemize}
\end{quest}

\begin{quest}[2016 A3]
Prove that $X^3+X+1$ is irreducible in $\Q[X]$. (You may use any general result that you wish, provided you state it clearly.)
\end{quest}

\begin{quest}[2015 B6 (Aug/May), 2018 B8]
Suppose that $R$ is a unique factorisation domain, with fraction field $K$. 
\begin{itemize}
\item Define the notion of \textit{content} $c(f)$ of an element $f\in R[X]$ and prove that $c(fg)=c(f)c(g)$ for $g\in R[X]$.
\item Show that if $f$ is an irreducible element of $R[X]$ that is irreducible in $R[X]$, then it is irreducible in $K[X]$.
\item Show that $X^3-5X+1$ and $X^3-3X+1$ are irreducible in $\Q[X]$.
\end{itemize}
\end{quest}

\begin{quest}[2017 B6]
Give an example of a principal ideal domain $A$ such that $A[X]$ is also a principal ideal domain, and of a principal ideal domain $B$ such that $B[X]$ is not a principal ideal domain. Justify your answers.
\end{quest}

\begin{quest}[2017 B7]
Suppose that $A$ is a unique factorisation domain. Say what it means for an element $f\in A[X]$ to be \textit{primitive}, and that the product of two primitive polynomials is primitive.
\end{quest}

\begin{quest}[2017 B8]
Consider the homomorphism $\phi: \Q[X,Y] \to \Q[t]$ given by $\phi(X)=t^3$, $\phi(Y)=t^4$.
\begin{itemize}
\item Find an element $f$ of $\text{ker}(\phi)$ such that $\text{deg}(f)=4$.
\item Show that $f$ generates $\text{ker}(\phi)$ as an ideal of $\Q[X,Y]$.
\end{itemize}
\end{quest}

\begin{quest}[2016 B7]
Let $A$ denote the subring $\Q[t^2,t^5]$ of the polynomial ring $\Q[t]$. Determine the kernel of the homomorphism $\phi: \Q[X,Y]\to A$ given by $\phi(X)=t^2$, $\phi(Y)=t^5$.
\end{quest}

\begin{quest}[2015 B7 (Aug/May), 2016 B8]
Suppose that $A$ is a ring, $I$ an ideal of $A$, $M$ an $A$-module with $n$ genrators and $\phi: M\to M$ an $A$-homomorphism such that $\phi(M)\subseteq IM$. Show that $\phi$ satisfies an equation of the form
\begin{equation*}
\phi^n + a_{n-1}\phi^{n-1} + \cdots +a_0 = 0
\end{equation*}
where $a_{n-1},\ldots , a_0\in I$.
\end{quest}

\begin{quest}[2015 B8 (Aug)]
If $R,S$ are rings, then the direct sum $R\oplus S$ is the ring of ordered pairs $(r,s)$ with $r\in R$ and $s\in S$ with the ring operations $(r,s)(r',s')=(rr',ss')$ and $(r,s)+(r',s')=(r+r', s+s')$. Show that if $I,J$ are ideals in a ring $A$ such that $I+J=A$, then $A/(I\cap J)=(A/I)\oplus (A/J)$.
\end{quest}

\begin{quest}[2015 B5 (May)]
Suppose that $k$ is a field, that the characteristic of $k$ is not $2$ and that $X,Y,Z,S,T$ are independent indeterminates. Consider the homomorphism
\begin{equation*}
\phi: k[X,Y,Z] \to k[S,T]
\end{equation*}
defined by $X\mapsto S^2$, $Y\mapsto ST$, $Z\mapsto T^2$. Describe $\text{ker}(\phi)$ in terms of a generator or generators.
\end{quest}



\end{document}